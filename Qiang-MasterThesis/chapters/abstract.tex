
\begin{abstract}

网络上涌现的海量图像为图像检索(Image Retrieval)领域带来了新的挑战。相似图像(Similar Image)是指在视觉上具有相似性,即一致性的区域经过各种变换的图片。由于相似图像之间只有部分区域相似,并且这些区域具有非全局的变换,如旋转、遮挡、视角、亮度、尺度等,这种条件下,如颜色、纹理等全局特征明显不适用于不具备全局变换的相似图像检索。目前,针对相似图像检索问题,主要有两种解决方案。传统方案是在图像中提取局部特征(SIFT、SURF、ORB等),构建索引(Bag-of-Words)进行图像相似度的计算并排序;第二是使用深度学习的解决方案。深度卷积网络(CNN)在计算机视觉领域取得了巨大的成功,如图像分类、目标检测、人脸识别等。对于图像检索问题,一般是使用训练好的模型(同时可能会在某些数据集上进行fine-tune)提取图片的全局特征(网络某一层的输出),然后进行相似度的计算并排序。

本论文主要对图像检索中的几何校验算法进行研究。基于图片局部特征的方案主要使用Bag-of-Words(BoW)构建索引,但是BoW不考虑特征点之间的几何一致性,导致图像之间会产生大量的误匹配,严重影响检索的性能。针对这个问题,本文提出Region Similarity Arrangement(RSA)的方法对局部特征点进行几何校验。不同于传统的几何校验方法,关注特征点之间位置关系或角度和尺度的差值,RSA构建一个区域属性空间(Region Property Space),并将每一个特征区域映射到该空间上,然后将该空间中点的分布编码到BoW向量中,完成对特征点之间的几何校验。虽然基于CNN的检索方案提取图片的全局特征,但是目前并没有很好的解决图片旋转的问题。对于这个问题,我们首先训练一个用于判断图片主方向的网络,然后对输入图片进行预处理,在此基础之上进行图片检索。

\keywords{社会影响力,\quad{}小世界网络,\quad{}近似算法,\quad{}路由}

\end{abstract}

\begin{englishabstract}

Influence maximization is the problem of selecting $k$ nodes in a social network to maximize their influence spread.
The problem has been extensively studied but most work focuses on the submodular influence diffusion models.
In this paper, motivated by empirical evidences, we explore influence maximization in the non-submodular regime.
In particular, we study the general threshold model in which a fraction of nodes have non-submodular threshold
	functions, but their threshold functions are closely upper- and lower-bounded by some submodular
	functions (we call them $\varepsilon$-almost submodular).
We first show a strong hardness result: there is no $1/n^{\frac{\gamma}{c}}$ approximation for influence maximization (unless P = NP)
	for all networks with up to $n^{\gamma}$ $\varepsilon$-almost submodular nodes, where $\gamma$ in $(0,1)$ 
	and $c$ is parameter depending on $\varepsilon$.
We then provide constant approximation algorithms when the number of non-submodular nodes are constant.
Finally, we conduct experiments on a number of real-world datasets, and the results demonstrate that our approximation algorithms
	outperform other benchmark algorithms.

\englishkeywords{Social Influence, Small World Network, Approximation Algorithm, Routing}

\end{englishabstract}
