
\chapter{绪论}
二十一世纪互联网发展迅猛,人与人之间不再仅仅只能当面交流,电子邮件、电话和即时通讯软件让人们之间的沟通更迅捷。与此同时,社交网站也在兴起,积累了越来越多的用户。在社交网络中,信息传播得更快,在社交网络中的营销手段会比传统营销获得更大的收益,有一些相关学者开始关注社交网络中的信息传播现象。


\section{课题研究的背景与意义}
社交网络(Social network)是由许多节点构成的一种社会结构。节点通常是组织和个人,节点之间的连接代表社会个体之间的关系,经由这些社会关系,把从偶然相识的泛泛之交到紧密结合的家庭关系的各种人们或组织串连起来。“社交网络”的概念从心理学、社会学、人类学、数学、统计学、计算机科学等不同领域不断深化,形成了一套系统的理论、方法和技术。在21世纪,人们获取信息的途径不再局限于报纸、广播、电视和当面交谈,随着Facebook,Twitter和Weibo等社交工具的广泛应用,社交网络已经成为重要的信息传播工具。在社交网络上,人们通过添加好友和关注建立人与人之间的连接,通过发消息、短文章、分享连接进行交流。信息在社交网络上可以沿着人与人之间的连接很快传播,很多新闻、广告在社交网络上可以很快的覆盖到绝大多数用户,这相对于传统的媒体介质有很大的优势。在互联网时代,社交网络成为了重要的信息传播工具,利用社交网络进行产品或信息的推广是很有力的手段。在实际应用场景中,会有这样的案例,某公司准备发售一款新产品,想要在社交网络上做一些免费体验活动,希望免费体验的人可以把产品较好的口碑在朋友间扩散出去,最后达到产品推广的目的。随之产生了这样一个优化问题,给定了网络结构和信息传播的模型之后,如何选定k个人作为种子,使得这$k$个人最后影响的范围最大,这就是社会网络影响力最大化\cite{Kempe2003maximizing}。这个问题被证明是NP难的,而且因为社交网络的规模很大,暴力的枚举所有种子集合很不现实,设计高效的有近似比保证的算法是很有需求的。

社会网络影响力最大化在被Kempe, Kleinberg和Tardos\cite{Kempe2003maximizing}提出并公式化之后,已经在学术界引起了广泛的关注,近年来很多学者都在做相关研究,涉及到病毒营销、媒体广告和谣言传播等方向。很多学者提出了比较有效的算法\cite{Kempe2003maximizing,Leskovec2007celf,Chen2009efficient,chen2010sharpphard,tang2014newrrset},这些算法大部分是采用了独立级联(Independent Cascade)或者线性阈值(Linear Threshold)[1]的信息传播模型。在这两种传播模型下,影响力函数是次模(Submodular)的,这时候可以使用贪心策略得到1-1/e近似的算法。此外,在一个更一般的通用阈值(General Threshold)\cite{Kempe2003maximizing}传播模型下,每个用户的阈值函数(Threshold Function)不再是简单的把边的权值相加,而是一个集合函数。伯克利的学者Elchanan和Sebastien\cite{Mossel2007sub}证明了阈值函数是次模的时候整体的影响力函数也是次模的,也就意味着局部的次模性质会导致全局的影响力函数的次模性。

然而,在真实的社交网络中,非次模的影响力传播现象经常出现。Backstrom[6]研究了LiveJounral和DBLP两个大型的社交网络数据,他绘制了个体加入某个社区的意愿与她的已经加入该社区好友数量的关系图。论文的图中可以看到意愿曲线整体上是上凸的,但是在最开始几个点有明显的下移。杨洋[7]等人观测了另一个社交网络Flickr,他们主要观察人的情绪变化被他带有情绪的朋友的影响。他们指出人变快乐的可能性和已经快乐的好友数量是一个超线性关系,尤其是那些影响力比较大的朋友。这些结果都指出现有的基于独立级联或者线性阈值模型的算法在真实的社会网络种可能并不能使用。很多非次模的传播模型已经被证明很难近似,像谣言和疾病的传播,需要社交网络中的个体在被影响邻居的数量超过某个阈值的时候才会被影响,这种模型也是通用阈值模型的变种,被称作固定阈值模型(Fixed Threshold Model)。在固定阈值模型下,社交网络影响力最大化问题是NP-hard的,而且有很强的不可近似性[1]。与此同时,如果考虑寻找能达到给定影响力目标的最小种子集合问题,也就是社会影响力最小化种子集合问题,陈宁也证明了这个问题也很难被近似[8]。学者们倾向于相信是次模性质帮助我们在社会影响力最大化问题中找到了比较好的近似算法。但是我们可以使用次模性质到什么程度呢?如果影响力传播过程仅仅是稍微偏离了次模性质,那么是否还有可能仍然设计一个近似比足够好的算法呢?这些问题现在仍然是需要解决的。

与此同时,固定阈值模型下的问题还有一些仍未被探索,前面说到的影响力传播,实际上只关心最后影响的人的数量,而不关心传播的时间和步数。考虑给定种子和目标节点的情况下,在不知道整体网络结构时,如何才能通过影响最少的人而影响目标?其实可以规定一个时间片只能影响一个人,问题就变成了社会网络里面,在固定阈值模型下,从给定的种子到目标点需要经过多少步跳转或者多少个时间片。这就是从另一个角度来研究社会影响力,从影响力传播的时间而不是影响的范围。此外,这里研究的是路由现象,就像IP包在路由网络里转发一样,IP包只知道最后的目的地,并不知道每一步应该具体怎么转发,也不能在每一个路由器群发,只能一步一步的跳转。我们研究的就是在社交网络里面的路由现象,跟传统路由的区别有两个,一个是社交网络的结构,没有路由表;另一个是每个节点被影响的邻居超过一定阈值时才能被影响到。在文章里,称类似IP包跳转的现象为简单路由,把影响需要阈值的路由为复杂路由。

社交网络是一个小世界网络,人与人之间的最短路很短。二十世纪60年代,美国哈佛大学社会心理学家斯坦利·米尔格伦(Stanley Milgram)做了一个连锁信实验[9]。他将一些信件交给自愿的参加者,要求他们通过自己的熟人将信传到信封上指明的收信人手里,他发现,294封信件中有64封最终送到了目标人物手中。而在成功传递的信件中,平均只需要5次转发,就能够到达目标。也就是说,在社会网络中,任意两个人之间的“距离”是6。这就是所谓的“六度分隔”理论(Six Degrees of Separation)。尽管他的实验有不少缺陷,但这个现象引起了学界的注意。小世界网络就是对这种现象(也称为小世界现象)的数学描述。用数学中图论的语言来说,小世界网络就是一个由大量节点构成的图,其中任意两点之间的平均路径长度比节点数量小得多。除了社会网络以外,小世界网络的例子在生物学、物理学、计算机科学等领域也有出现。许多现实中的图可以由小世界网络作为模型。万维网、公路交通网、脑神经网络和基因网络都呈现小世界网络的特征。本文采用经典的Kleinberg网络[10],这是一个基于二维网格的小世界网络,网格的边被称为人与人之间的强链接,而同时每个节点会发出若干条随机的弱连接。Kleinberg指出当模型的参数α等于网格的维度时,贪心路由算法有很高的效率。这里的贪心路由算法是指在每一步,当前节点把消息传递给他的邻居里面距离目标节点曼哈顿距离最小的节点。这个路由算法的需要的跳转数也符合之前Milgram的实验结果,从理论上支持了小世界现象。Kleinberg进一步指出当模型的参数α不等于网格的维度时,所有的路由算法都不能很快地把信件送达到目标手中。本文关注的是小世界网络中复杂路由的速度,复杂路由每一步激活的过程就是固定阈值模型,也是通用阈值模型范畴下的子问题。


\section{国内外研究现状}
自从基于BoW的图像检索框架\cite{sivic2003video}被提出以来,国内外学者对BoW中的几何校验问题就从未终止。针对BoW框架的几何校验基本可以分为两类,即检索后校验和检索时校验。检索后校验是指在进行正常的BoW检索之后,基于查询图片和匹配图片中特征点的几何一致性,对检索结果进行重新排序。检索后校验只考虑匹配点的几何关系,并且一般情况下,后校验的算法时间复杂度较高,只能对初排序结果的top N进行几何校验。检索时校验是指在进行BoW检索过程中进行几何校验,所以这种方法会考虑每对特征点之间的几何一致性。这就要求必须在进行检索之前将一些额外的有关于几何关系的信息融入到BoW向量中或倒排表内。一般情况下,检索时校验不会进行非常严格的几何校验,但是具有较高的检索效率。

\cite{sivic2003video}使用了一种相对简单的几何校验方案,称为空间一致性(Spatial Consistency)校验。这种校验考虑每个匹配点(局部特征点或区域)最近邻的匹配情况。具有空间一致性的点其最近邻的匹配点也应该在同一块区域内。这种方法只考虑了特征点在图片中相对位置关系的一致性。可以看出空间一致性检验是一种后校验方案。

利用RANSAC算法进行空间校验可以得到最好的几何校验效果,因为它要求很高的几何变换的一致性。首先定义匹配点之间变换矩阵,一般会考虑平移、旋转和尺度变换;然后计算每对匹配点之间的变换矩阵,并基于RANSAC算法得到最优的变化矩阵;最后根据这个变换矩阵,测试每对匹配点,符合这个变换的的点对被称为“inliers”,“inliers”数决定图片最终的相似度。\cite{philbin2007object}基于上述思想提出了Fast Spatial Matching(FSM)算法。由于FSM也是一种后校验的算法,FSM具有较高的计算复杂度,并且只能对Top N的检索结果进行校验,不能保证高召回率。\cite{Zhong2015Fast}对FSM做出改进并提出Direct Spatial Matching(DSM)。DSM直接计算尺度的变换,相比于其他基于RANSAC的算法,DSM需要更少的校验时间。

由于后校验的方法计算复杂度高、低召回率等问题,检索时校验同样得到人们的关注。这其中最经典的方法是基于空间金字塔(Spatial Pyramid)的几何校验算法。\cite{lazebnik2006beyond}和\cite{Cao2010Spatial}都是基于某种规则将一副图像分成若干区域,然后为每个区域生成一个BoW向量,最后将这些BoW向量连接到一起生成整幅图像的BoW向量。\cite{Cao2010Spatial}提出Spatial Bag-of-Features是\cite{lazebnik2006beyond}中Spatial Pyramid Matching的一般形式并且Spatial Bag-of-Features可以处理目标的基本变换,例如平移、旋转和尺度变化等。

近些年,CNN(Convolutional Neural Networks)在计算机视觉领域取得了巨大的成功,例如图像分类\cite{krizhevsky2012imagenet}、目标检测\cite{Girshick2014Rich}\cite{girshick2015fast}\cite{ren2015faster}、人脸识别,也包括图像检索\cite{babenko2014neural}。对于图像检索问题,一般是使用训练好的模型(同时可能会在某些数据集上进行fine-tune)提取图片的全局特征(网络某一层的输出),然后进行相似度的计算并排序。不同于传统的BoW检索框架,基于CNN的检索框架不考虑图片的旋转和尺度变化,而完全依赖于模型的能力和大规模的训练数据。在实际检索场景下,图片旋转是常见问题。现有网络一般使用Max Pooling操作,可以一定程度上应对旋转问题。

可以看到,国际上对于图像检索中几何校验的研究正呈现百花齐放的状态,研究人员从不同角度对几何校验进行了多方面的研究,形成了不同的算法与思路,这也恰好说明了人们对于显著性区域检测这一课题还处于探索研究阶段,并没有形成最优的解决方案,还存在多方面的问题亟待解决。



\section{本文所做的工作}
本文针对目前检索框架(包括BoW检索框架和CNN检索框架)中存在的问题,在Word Spatial Arrangement(WSA)\cite{penatti2014visual}基础上做出改进,提出Region Property Arrangement(RSA)空间校验算法。并且训练了学习图片主方向的CNN网络Main Orientation Net来解决CNN检索框架中图片旋转的问题。主要工作与成果如下:

\begin{enumerate}
\item  通过对WSA和其他空间校验算法的深入研究,提出了RSA空间校验算法。RSA方法首次提出了Region Property Space(RPS)的概念,并将图片中的特征区域(interest region)映射为RPS中的一个点。传统的空间校验算法只考虑匹配点对之间尺度、角度的相对变化,而RSA可以分析一副图片所有特征区域属性的分布规律,并将这种分布编码到BoW向量中,可以在检索过程中完成几何校验。RSA算法可以显著提高BoW检索的性能,在Holidays、Paris和Oxford数据库上达到了state-of-the-art的检索性能,同时不会增加计算和存储的负担。
\item  通过对RPS中点的分布进行深入的研究,我们发现RPS中点的分布具有很强的规律性,这种规律性在图片位置空间(基于特征点在图片上出现的位置)所不具备的。通过分析,我们在RPS的基础上提出了Spatial Weighting(SpW)。SpW可以解决图像检索中的burstniess\cite{jegou2009burstiness}的问题。并且,基于RPS中点的分布特性,我们提出了计算RSA的快速算法,比暴力算法快2到3倍。
\item 为了应对图像检索中图片旋转的问题,我们提出了MONet(Main Orientation Net)。由于图片主方向包含了图片的语义信息,传统的方法无法得到(只能得到某个区域的梯度方向)。借助深度学习方案,MONet很好的解决了查询图像可以旋转的问题,提高的检索的准确率。
\end{enumerate}

\section{论文组织结构}
本文分为五章,主要结构和内容如下:

第一章首先阐述了显著几何校验的研究背景和意义,接着介绍了该方向的国内外研究现状以及存在的问题,最后概述了本文所做的工作和论文的组织结构。

第二章介绍显几何校验测的基础特征与算法,并分类探讨了目前国际上各类主流算法的思路与不足。基于此,制定了本文的研究思路与框架。

第三章介绍Region Similarity Arrangement和Spatial Weighting算法,并通过详细的实验对比,全面评测了该算法的性能。

第四章介绍 Main Orientation Net,并将该网络应用到BoW和CNN检索框架中,对算法性能进行了全面评估。

最后一章对全文进行了总结,同时展望未来的研究工作。
