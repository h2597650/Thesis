
\chapter{绪论}
二十一世纪互联网发展迅猛,人与人之间不再仅仅只能当面交流,电子邮件、电话和即时通讯软件让人们之间的沟通更迅捷。与此同时,社交网站也在兴起,积累了越来越多的用户。在社交网络中,信息传播得更快,在社交网络中的营销手段会比传统营销获得更大的收益,有一些相关学者开始关注社交网络中的信息传播现象。


\section{课题研究的背景与意义}
社交网络(Social network)是由许多节点构成的一种社会结构。节点通常是组织和个人,节点之间的连接代表社会个体之间的关系,经由这些社会关系,把从偶然相识的泛泛之交到紧密结合的家庭关系的各种人们或组织串连起来。“社交网络”的概念从心理学、社会学、人类学、数学、统计学、计算机科学等不同领域不断深化,形成了一套系统的理论、方法和技术。在21世纪,人们获取信息的途径不再局限于报纸、广播、电视和当面交谈,随着Facebook,Twitter和Weibo等社交工具的广泛应用,社交网络已经成为重要的信息传播工具。在社交网络上,人们通过添加好友和关注建立人与人之间的连接,通过发消息、短文章、分享连接进行交流。信息在社交网络上可以沿着人与人之间的连接很快传播,很多新闻、广告在社交网络上可以很快的覆盖到绝大多数用户,这相对于传统的媒体介质有很大的优势。在互联网时代,社交网络成为了重要的信息传播工具,利用社交网络进行产品或信息的推广是很有力的手段。在实际应用场景中,会有这样的案例,某公司准备发售一款新产品,想要在社交网络上做一些免费体验活动,希望免费体验的人可以把产品较好的口碑在朋友间扩散出去,最后达到产品推广的目的。随之产生了这样一个优化问题,给定了网络结构和信息传播的模型之后,如何选定k个人作为种子,使得这$k$个人最后影响的范围最大,这就是社会网络影响力最大化\cite{Kempe2003maximizing}。这个问题被证明是NP难的,而且因为社交网络的规模很大,暴力的枚举所有种子集合很不现实,设计高效的有近似比保证的算法是很有需求的。

社会网络影响力最大化在被Kempe, Kleinberg和Tardos\cite{Kempe2003maximizing}提出并公式化之后,已经在学术界引起了广泛的关注,近年来很多学者都在做相关研究,涉及到病毒营销、媒体广告和谣言传播等方向。很多学者提出了比较有效的算法\cite{Kempe2003maximizing,Leskovec2007celf,Chen2009efficient,chen2010sharpphard,tang2014newrrset},这些算法大部分是采用了独立级联(Independent Cascade)或者线性阈值(Linear Threshold)[1]的信息传播模型。在这两种传播模型下,影响力函数是次模(Submodular)的,这时候可以使用贪心策略得到1-1/e近似的算法。此外,在一个更一般的通用阈值(General Threshold)\cite{Kempe2003maximizing}传播模型下,每个用户的阈值函数(Threshold Function)不再是简单的把边的权值相加,而是一个集合函数。伯克利的学者Elchanan和Sebastien\cite{Mossel2007sub}证明了阈值函数是次模的时候整体的影响力函数也是次模的,也就意味着局部的次模性质会导致全局的影响力函数的次模性。

然而,在真实的社交网络中,非次模的影响力传播现象经常出现。Backstrom\cite{backstrom2006group}研究了LiveJounral和DBLP两个大型的社交网络数据,他绘制了个体加入某个社区的意愿与她的已经加入该社区好友数量的关系图。论文的图中可以看到意愿曲线整体上是上凸的,但是在最开始几个点有明显的下移。杨洋\cite{yang2016role}等人观测了另一个社交网络Flickr,他们主要观察人的情绪变化被他带有情绪的朋友的影响。他们指出人变快乐的可能性和已经快乐的好友数量是一个超线性关系,尤其是那些影响力比较大的朋友。这些结果都指出现有的基于独立级联或者线性阈值模型的算法在真实的社会网络种可能并不能使用。很多非次模的传播模型已经被证明很难近似,像谣言和疾病的传播,需要社交网络中的个体在被影响邻居的数量超过某个阈值的时候才会被影响,这种模型也是通用阈值模型的变种,被称作固定阈值模型(Fixed Threshold Model)。在固定阈值模型下,社交网络影响力最大化问题是NP-hard的,而且有很强的不可近似性\cite{Kempe2003maximizing}。与此同时,如果考虑寻找能达到给定影响力目标的最小种子集合问题,也就是社会影响力最小化种子集合问题,陈宁也证明了这个问题也很难被近似\cite{Chen2008approximability}。学者们倾向于相信是次模性质帮助我们在社会影响力最大化问题中找到了比较好的近似算法。但是我们可以使用次模性质到什么程度呢?如果影响力传播过程仅仅是稍微偏离了次模性质,那么是否还有可能仍然设计一个近似比足够好的算法呢?这些问题现在仍然是需要解决的。

与此同时,固定阈值模型下的问题还有一些仍未被探索,前面说到的影响力传播,实际上只关心最后影响的人的数量,而不关心传播的时间和步数。考虑给定种子和目标节点的情况下,在不知道整体网络结构时,如何才能通过影响最少的人而影响目标?其实可以规定一个时间片只能影响一个人,问题就变成了社会网络里面,在固定阈值模型下,从给定的种子到目标点需要经过多少步跳转或者多少个时间片。这就是从另一个角度来研究社会影响力,从影响力传播的时间而不是影响的范围。此外,这里研究的是路由现象,就像IP包在路由网络里转发一样,IP包只知道最后的目的地,并不知道每一步应该具体怎么转发,也不能在每一个路由器群发,只能一步一步的跳转。我们研究的就是在社交网络里面的路由现象,跟传统路由的区别有两个,一个是社交网络的结构,没有路由表;另一个是每个节点被影响的邻居超过一定阈值时才能被影响到。在文章里,称类似IP包跳转的现象为简单路由,把影响需要阈值的路由为复杂路由。

社交网络是一个小世界网络,人与人之间的最短路很短。二十世纪60年代,美国哈佛大学社会心理学家斯坦利·米尔格伦(Stanley Milgram)做了一个连锁信实验\cite{Milgram1967small}。他将一些信件交给自愿的参加者,要求他们通过自己的熟人将信传到信封上指明的收信人手里,他发现,294封信件中有64封最终送到了目标人物手中。而在成功传递的信件中,平均只需要5次转发,就能够到达目标。也就是说,在社会网络中,任意两个人之间的“距离”是6。这就是所谓的“六度分隔”理论(Six Degrees of Separation)。尽管他的实验有不少缺陷,但这个现象引起了学界的注意。小世界网络就是对这种现象(也称为小世界现象)的数学描述。用数学中图论的语言来说,小世界网络就是一个由大量节点构成的图,其中任意两点之间的平均路径长度比节点数量小得多。除了社会网络以外,小世界网络的例子在生物学、物理学、计算机科学等领域也有出现。许多现实中的图可以由小世界网络作为模型。万维网、公路交通网、脑神经网络和基因网络都呈现小世界网络的特征。本文采用经典的Kleinberg网络\cite{Kleinberg2000small},这是一个基于二维网格的小世界网络,网格的边被称为人与人之间的强链接,而同时每个节点会发出若干条随机的弱连接。Kleinberg指出当模型的参数$\alpha$等于网格的维度时,贪心路由算法有很高的效率。这里的贪心路由算法是指在每一步,当前节点把消息传递给他的邻居里面距离目标节点曼哈顿距离最小的节点。这个路由算法的需要的跳转数也符合之前Milgram的实验结果,从理论上支持了小世界现象。Kleinberg进一步指出当模型的参数$\alpha$不等于网格的维度时,所有的路由算法都不能很快地把信件送达到目标手中。本文关注的是小世界网络中复杂路由的速度,复杂路由每一步激活的过程就是固定阈值模型,也是通用阈值模型范畴下的子问题。


\section{国内外研究现状}
在Kempe, Kleinberg和Tardos\cite{Kempe2003maximizing}提出影响力最大化的问题之后,近些年来很多工作陆续研究了这一问题\cite{bharathi2007competitive,Leskovec2007celf,Chen2009efficient,chen2010sharpphard,Goyal2011simpath}。其中值得注意的是Leskovec\cite{Leskovec2007celf}在07年提出了一种lazy-forward的优化方法,利用了影响力函数$\sigma(\cdot)$的次模性质,本轮计算的$\Delta_x\sigma(\cdot)$随着轮次增加只会递减而无需再次计算,避免了对种子集合影响力的重复计算,极大地提升了贪心算法的速度。陈卫在09年\cite{Chen2009efficient}和10年\cite{chen2010sharpphard}连续提出了高性能的启发式算法,可以在DBLP等有六十万节点百万条边的大型网络上运行,而且算法效果几乎和贪心算法持平。算法的主要思想是把影响传播过程近似分解为影响力路径,计算每条影响力路径的概率,最后近似得到影响力大小。在14年,Borgs等人\cite{borgs2014rrset}提出了逆向可达集合(Reverse Reachable Set)的技术来加速计算一个集合影响某个节点的概率,然后把问题转化为最大集合覆盖问题,并且依靠这个技术首次把影响力最大化算法的复杂度降到近线性。Tang等人\cite{tang2014newrrset}工程化的实现了Borgs提出的近线性算法并且在具有十亿条边的Twitter网络上测试了算法的性能和效率。紧接着,Tang\cite{tang2015influence}和Nguyen\cite{mtai2016sigmod}进一步提升了基于逆向可达集合的贪心算法的效果,主要是改进了分析的办法,利用更少的逆向可达集合来保证算法的运行。这一系列的影响力最大化相关工作大部分都是利用了次模性质来加速了近似算法,得到$1-1/e$的近似比。

种子集合最小化是和影响力最大化对立的问题,是找到一个影响力超过给定阈值的最小种子集合的问题。陈宁在08年\cite{Chen2008approximability}给出了固定阈值模型下很强的不可近似结果。固定阈值模型也是通用阈值模型的特例,固定阈值模型的阈值函数是在阈值处有一个从0到1的突变。Goyal在12年\cite{goyal2012minimizing}提出了一个双向近似的种子集合最小化的贪心算法。最近,张鹏\cite{zhang2014prob}在KDD14上第一次提出了种子集合最小化的一个概率变种模型,主要是关注给定的种子集合的影响力能否以一定的概率保证达到给定阈值,而之前的工作大部分只关心最终影响力的期望的大小。


非次模问题的最优化是近些年研究的另一个关注点,次模问题的最优化已经有比较成熟的结果,非次模问题还有很多需要探索。堵丁柱\cite{du2008analysis}在08年提出了约束次模和偏移次模两个技术用来研究非次模优化的贪心算法近似比,并且在Steiner Tree问题上给了细致的分析。Horel等人\cite{Horel2016sub}刚刚在NIPS16上研究了一类非次模但是和次模函数很接近的函数优化问题,他们假设函数值可以从一个数据库里面不断获取,关注的是找到函数h最大值点的同时需要从数据库询问的次数。

对于Kleinberg小世界网络[10],Nguyen和Martel在2004年\cite{Martel2004analyzing}和2005年\cite{Nguyen2005analyzing}分析了小世界网络中简单传播时间的上下界,也就是小世界网络的半径。他们指出对于二维网格在$\alpha<4$时,Kleinberg’s model都是一个小世界网络,网络中的任意两个节点之间的最短路径是网络中节点个数的对数级。Ghasemiesfeh等人\cite{Ghasemiesfeh2013complex}在2013年首次给出了复杂传播的理论分析,他们研究了一个节点只有在至少$k$个邻居被激活时才能激活的$k$-复杂传播。他们的理论分析支持了弱连接在复杂传播过程中作用较小的观点,同时指出复杂传播的速度很大程度上取决于弱连接的分布,在15年\cite{ebrahimi2015complex}他们进一步探索了复杂传播和聚集系数$\alpha$的关系。Gao等人在16年\cite{gao2016gt}进一步研究了阈值模型下每个节点的阈值服从一个分布时$k$-复杂传播的一些问题。




\section{本文所做的工作}
1、研究内容
社交网络影响力传播的非次模现象很普遍,准备研究非次模问题里一个比较特殊的例子并且深入下去,从通用阈值模型入手。通用阈值模型在每个节点的阈值函数都是次模的时候,影响力函数也是次模的[5]。我准备研究当阈值函数不是次模,但又和次模函数很接近的情况。我定义了ε次模逼近函数,这个集合函数不是次模但是被两个相差ε的次模函数夹住。阈值函数是ε次模逼近的情况下,模型相当于是一个次模的模型偏移了一点,希望可以在这个不是次模的模型上仍然做出一些正面的结果,比如给出一些近似比不错的算法或者证明问题是可以近似的。或者虽然新模型和次模模型很接近,但是因为不是次模,可以给出不可近似性的结果。最后可以设计实验,测试一下在非次模模型上影响力算法的实验结果。

在小世界网络的复杂路由方面,也是希望要么给出比较好的路由算法并且证明路由步数是对数量级的,要么就给出路由时间的下界。Kleinberg[10]和Ghasemiesfeh等[23,24]已经证明了小世界网络里简单传播、复杂传播和简单路由都是很快速的,我们的结果会补上最后一个复杂路由的缺口。


\begin{enumerate}
\item  通过对WSA和其他空间校验算法的深入研究,提出了RSA空间校验算法。RSA方法首次提出了Region Property Space(RPS)的概念,并将图片中的特征区域(interest region)映射为RPS中的一个点。传统的空间校验算法只考虑匹配点对之间尺度、角度的相对变化,而RSA可以分析一副图片所有特征区域属性的分布规律,并将这种分布编码到BoW向量中,可以在检索过程中完成几何校验。RSA算法可以显著提高BoW检索的性能,在Holidays、Paris和Oxford数据库上达到了state-of-the-art的检索性能,同时不会增加计算和存储的负担。
\item  通过对RPS中点的分布进行深入的研究,我们发现RPS中点的分布具有很强的规律性,这种规律性在图片位置空间(基于特征点在图片上出现的位置)所不具备的。通过分析,我们在RPS的基础上提出了Spatial Weighting(SpW)。SpW可以解决图像检索中的burstniess\cite{jegou2009burstiness}的问题。并且,基于RPS中点的分布特性,我们提出了计算RSA的快速算法,比暴力算法快2到3倍。
\item 为了应对图像检索中图片旋转的问题,我们提出了MONet(Main Orientation Net)。由于图片主方向包含了图片的语义信息,传统的方法无法得到(只能得到某个区域的梯度方向)。借助深度学习方案,MONet很好的解决了查询图像可以旋转的问题,提高的检索的准确率。
\end{enumerate}

\section{论文组织结构}
本文主要是研究Kleinberg小世界网络中复杂传染病的传播时间和路由时间,
首先分析了小世界模型的特性,包括已有的关于简单传播(直径)和简单路由的结论。
然后,对相关小世界模型和复杂传染病模型的定义做出详细论述。
其次,给出了有关小世界网络上传播和路由的蒙特卡洛模拟实验的完整过程,包括设计、步骤和结果分析等。
接下来给出有关复杂传播和复杂的主要结论以及完整证明。
最后给出本文结论和已有结论的分析对比。

论文组织结构如下:

第一章 绪论

介绍课题的研究背景,阐述了小世界现象的研究背景和以及国内外研究现状,提出本文的研究目标和主要工作内容。

第二章 相关技术综述

介绍本文工作涉及到的主要相关理论及模型,重点分析了Kleinberg小世界网络以及复杂传染病的相关知识,
然后介绍了蒙特卡洛模拟的相关技术。

第三章 蒙特卡洛模拟及实验结果分析

首先给出了基于Kleinberg小世界网络的蒙特卡洛模拟实验的设计,然后介绍实验目的和步骤,
最后给出实验结果和分析过程。

第四章 复杂传播的研究

首先给出复杂路由的主要结论,然后给出证明的框架和基于框架给出完整的证明。
最后把结论扩展到$k$-复杂传播并与简单传播进行比较。


第五章 复杂路由的研究

给出复杂路由的结论及完整的证明过程,然后探讨一步允许感染$m$个节点的复杂路由,建立起复杂传播和复杂路由的关系。

第六章 总结与展望

在最后的总结与展望中,首先整体给出了文章的结论,然后对比本文结果和已有结果,深入讨论了复杂传播和简单传播的内在差别。
然后指出已有工作的不足,并对下一步的工作进行展望。