
\chapter{绪论}
二十一世纪互联网发展迅猛,人与人之间不再仅仅局限于当面交流,
电子邮件、电话和即时通讯软件让人们之间的沟通更迅捷。与此同时,社交网站也在兴起,积累了越来越多的用户。
在社交网络中,信息传播得更快,在社交网络中的营销手段会比传统营销获得更大的收益。
探索社交网络的信息传播可以帮助商家选择更好的营销策略,越来越多的学者开始关注社交网络中的信息传播现象。


\section{课题研究的背景与意义}
社交网络(Social network)是由许多节点构成的一种社会结构。节点通常是组织和个人,节点之间的连接代表社会个体之间的关系,经由这些社会关系,把从偶然相识的泛泛之交到紧密结合的家庭关系的各种人们或组织串连起来。“社交网络”的概念从心理学、社会学、人类学、数学、统计学、计算机科学等不同领域不断深化,形成了一套系统的理论、方法和技术。在21世纪,人们获取信息的途径不再局限于报纸、广播、电视和当面交谈,随着Facebook,Twitter和Weibo等社交工具的广泛应用,社交网络已经成为重要的信息传播工具。在社交网络上,人们通过添加好友和关注建立人与人之间的连接,通过发消息、短文章、分享连接进行交流。信息在社交网络上可以沿着人与人之间的连接很快传播,很多新闻、广告在社交网络上可以很快的覆盖到绝大多数用户,这相对于传统的媒体介质有很大的优势。在互联网时代,社交网络成为了重要的信息传播工具,利用社交网络进行产品或信息的推广是很有力的手段。在实际应用场景中,会有这样的案例,某公司准备发售一款新产品,想要在社交网络上做一些免费体验活动,希望免费体验的人可以把产品较好的口碑在朋友间扩散出去,最后达到产品推广的目的。随之产生了这样一类优化问题,给定了网络结构和信息传播的模型之后,如何选定$k$个人作为种子,使得这$k$个人最后影响的范围最大,这就是社会网络影响力最大化\cite{Kempe2003maximizing}。
在常见传播模型中,这个问题被证明是NP难的,而且因为社交网络的规模很大,暴力的枚举所有种子集合很不现实,设计高效的有近似比保证的算法是很有需求的。

社会网络影响力最大化在被Kempe, Kleinberg和Tardos\cite{Kempe2003maximizing}提出并公式化之后,已经在学术界引起了广泛的关注,近年来很多学者都在做相关研究,
涉及到病毒营销\cite{arthur2009pricing,Subramani2003knowledge,barbieri2014influence,Chen2015combining}、
媒体广告\cite{li2011labeled,bakshy2012advertising,Aslay2015viral}
和谣言传播\cite{carnes2007maximizing,kostka2008word,borodin2010competitive,he2012influence,Wang2016drimux}等方向。很多学者提出了比较有效的算法\cite{Kempe2003maximizing,Leskovec2007celf,Chen2009efficient,chen2010sharpphard,goyal2011celfplus,Goyal2011simpath,tang2014newrrset},
这些算法大部分是采用了独立级联(Independent Cascade)或者线性阈值(Linear Threshold)\cite{Kempe2003maximizing}的信息传播模型。
在这两种传播模型下,影响力函数是次模(Submodular)的,也就是边际收益递减(diminishing marginal returns)的集合函数。
这时候可以使用贪心策略得到$1-\frac{1}{e}$近似的算法。此外,在一个更一般的通用阈值(General Threshold)\cite{Kempe2003maximizing}传播模型下,每个用户的阈值函数(Threshold Function)不再是简单的把边的权值相加,而是一个集合函数。伯克利的学者Elchanan和Sebastien\cite{Mossel2007sub}证明了阈值函数是次模的时候整体的影响力函数也是次模的,也就意味着局部的次模性质会导致全局的影响力函数的次模性质。

然而,在真实的社交网络中,非次模的影响力传播现象经常出现。Backstrom等\cite{backstrom2006group}研究了LiveJounral和DBLP两个大型的社交网络数据,他们绘制了个体加入某个社区的意愿与他的已经加入该社区好友数量的关系图。论文的图中可以看到意愿曲线整体上是上凸的,但是在最开始几个点有明显的下移。Yang等人\cite{yang2016role}观测了另一个社交网络Flickr,他们主要观察人的情绪变化被他带有情绪的朋友的影响。他们指出人变快乐的可能性和已经快乐的好友数量是一个超线性关系,尤其是那些影响力比较大的朋友。这些结果都指出现有的基于独立级联或者线性阈值模型的算法在真实的社会网络种可能并不能使用。很多非次模的传播模型已经被证明很难近似,像谣言和疾病的传播,需要社交网络中的个体在被影响邻居的数量超过某个阈值的时候才会被影响,这种模型也是通用阈值模型的变种,被称作固定阈值模型(Fixed Threshold Model)。在固定阈值模型下,社交网络影响力最大化问题是NP难的,而且有很强的不可近似性\cite{Kempe2003maximizing}。
与此同时,如果考虑寻找能达到给定影响力目标的最小种子集合问题,也就是社会影响力最小化种子集合问题,
Ning Chen证明了这个问题也很难被近似\cite{Chen2008approximability}。学者们倾向于相信是次模性质帮助我们在社会影响力最大化问题中找到了比较好的近似算法。
但是我们可以使用次模性质到什么程度?
如果影响力传播过程仅仅是稍微偏离了次模性质,那么是否还有可能仍然设计一个近似比足够好的算法?这些问题现在仍然是需要解决的。

与此同时,固定阈值模型下的问题还有一些仍未被探索,前面说到的影响力传播,实际上只关心最后影响的人的数量,而不关心传播的时间和步数。考虑给定种子和目标节点的情况下,在不知道整体网络结构时,如何才能通过影响最少的人而影响目标?其实可以规定一个时间片只能影响一个人,问题就变成了社交网络里面,在固定阈值模型下,从给定的种子到目标点需要经过多少步跳转或者多少个时间片。这就是从另一个角度来研究社会影响力,从影响力传播的时间而不是影响的范围。此外,这里研究的是路由现象,就像IP包在路由网络里转发一样,IP包只知道最后的目的地,并不知道每一步应该具体怎么转发,也不能在每一个路由器群发,只能一步一步的跳转。我们研究的就是在社交网络里面的路由现象,跟传统路由的区别有两个,一个是社交网络的结构,没有路由表;另一个是每个节点被影响的邻居超过一定阈值时才能被影响到。在文章里,称类似IP包跳转的现象为简单路由,把影响需要阈值的路由为复杂路由。

社交网络是一个小世界网络,人与人之间的最短路很短。二十世纪60年代,美国哈佛大学社会心理学家斯坦利·米尔格伦(Stanley Milgram)做了一个连锁信实验\cite{Milgram1967small}。他将一些信件交给自愿的参加者,要求他们通过自己的熟人将信传到信封上指明的收信人手里,他发现,294封信件中有64封最终送到了目标人物手中。而在成功传递的信件中,平均只需要5次转发,就能够到达目标。也就是说,在社会网络中,任意两个人之间的“距离”是6。这就是所谓的“六度分隔”理论(Six Degrees of Separation)。尽管他的实验有不少缺陷,但这个现象引起了学界的注意。小世界网络就是对这种现象(也称为小世界现象)的数学描述。用数学中图论的语言来说,小世界网络就是一个由大量节点构成的图,其中任意两点之间的平均路径长度比节点数量小得多。除了社会网络以外,小世界网络的例子在生物学、物理学、计算机科学等领域也有出现。许多现实中的图可以由小世界网络作为模型。
除了社交网络以外,小世界现象在自然形成的或者工业推动产生的网络中都是很普遍的。
例如线虫的神经网络和美国西部的电力网\cite{Watts1998collective},
万维网中的超链接所构成的网络也具有小世界现象\cite{Albert1999internet}。

后续很多工作都在探索小世界现象。
Granovetter在他最原始的工作中\cite{Granovetter1973strength,Granovetter1974getting}
把社会关系分为{\it 强连接}和{\it 弱连接}。
强连接(Strong ties)指非常亲密的关系,例如家人、挚友等。
而相对于强连接,人们还维持着一些更广泛的社会关系,
例如偶然结识的新朋友、远处的同学,这些关系被称为弱连接(Weak ties)。
Granovetter通过调研麻省牛顿镇的居民如何找工作来探索社会网络\cite{Granovetter1974getting},
得到了一个令人惊讶的结论。
被调查的人中,大部分都是通过自己的人脉来找到现在的工作,
但是比较有趣的是这些人脉大部分都不是最亲密的那些朋友,
这被称为弱连接的力量\cite{Granovetter1973strength}。

Watts和Strogatz首先提出了基于圆环的小世界网络模型\cite{Watts1998collective},
模型按照如下步骤构造。
\begin{enumerate}
	\item 底层框架是一个圆环,由$n$个节点的集合$V$均匀分布在圆环上。
	\item 把所有节点与其最近的$k$个邻居相连,这里$k$是一个较小的常数,这些边被称作强连接。
	\item 对于这个网络中每条强连接,以$p$的概率重新选择边的终点,终点是完全随机地从$V$中选择的,重新连接的边被称作弱连接。
\end{enumerate}
这样就完成了小世界网络的建立。
他们还提出了小世界网络的应该具有的两个特性:直径较小(Short diameter)和聚集系数较大(High clustering coefficient)。
直径较小要求网络中任意两点之间存在比较短的路径。
聚集系数是一个节点的邻接点之间相互连接的程度,真实社会网络中,一个人的朋友之间一般都会相互认识,
所以真实网络聚集系数也比较大。
之前的研究\cite{de1978contacts}一般设定每个人都完全随机地从所有人选择一些人作为自己的朋友,而且相识关系是对称的。
这样构成的网络是一个随机图,而且的确有比较小的直径\cite{bollobas1998random}。
但是过于随机的网络导致这个网络的聚集系数比较小,这不符合社会网络的特性。
然而,如果一个网络的聚集系数比较大,例如一个二维网格,这个网络的直径会较大,没有小世界现象的存在。
Watts和Strogatz提出的这个模型是介于两个极端之间的模型,同时具有较小的直径和较大的聚集系数。
这个模型是一个圆环框架和随机图的叠加,圆环作为模型的底层结构(Underlying structure),而随机图代表网络中少量的弱连接。

Kleinberg改进了Watts和Strogatz的模型\cite{Kleinberg2000small},也是基于由强连接构成的框架和随机图的弱连接叠加而成。
Kleinberg的小世界网络的框架是一个二维的网格,网格上的边代表强连接。
同时网格上的每个节点$u$都会发出一条弱连接,而这条弱连接以$v$为终点的概率正比于
$1/|uv|^{\alpha}$,这里$|uv|$是节点$u$和节点$v$之间的网格距离,$\alpha$是小世界模型的参数(Clustering exponent)。
Kleinberg证明了当且仅当$\alpha$等于网格的维度(这里是$2$)时,贪心的分散式路由算法可以有效地把消息传递给指定的目标\cite{Kleinberg2000small}。
这里的分散式路由算法是因为在算法执行的每一步,算法只知道所有已经传递过消息的节点的邻居,而不清楚整个网络中弱连接的分布,可以理解为没有路由表。
而算法的贪心则是指在每一步,当前节点会选择距离目标网格距离最近的节点来传递消息。
$\alpha=2$的小世界网络中分散式路由的效率在数量级上符合Milgram的实验结果。



\section{国内外研究现状}
在Kempe, Kleinberg和Tardos\cite{Kempe2003maximizing}提出影响力最大化的问题之后,近些年来很多工作陆续研究了这一问题\cite{bharathi2007competitive,Leskovec2007celf,Chen2009efficient,chen2010sharpphard,Goyal2011simpath}。其中值得注意的是Leskovec\cite{Leskovec2007celf}在07年提出了一种lazy-forward的优化方法,利用了影响力函数$\sigma(\cdot)$的次模性质,本轮计算的$\Delta_x\sigma(\cdot)$随着轮次增加只会递减而无需再次计算,避免了对种子集合影响力的重复计算,极大地提升了贪心算法的速度。Chen等人在09年\cite{Chen2009efficient}和10年\cite{chen2010sharpphard}连续提出了高性能的启发式算法,可以在DBLP等有六十万节点百万条边的大型网络上运行,而且算法效果几乎和贪心算法持平。算法的主要思想是把影响传播过程近似分解为影响力路径,计算每条影响力路径的概率,最后近似得到影响力大小。在14年,Borgs等人\cite{borgs2014rrset}提出了逆向可达集合(Reverse Reachable Set)的技术来加速计算一个集合影响某个节点的概率,然后把问题转化为最大集合覆盖问题,并且依靠这个技术首次把影响力最大化算法的复杂度降到近线性。Tang等人\cite{tang2014newrrset}工程化的实现了Borgs提出的近线性算法并且在具有十亿条边的Twitter网络上测试了算法的性能和效率。紧接着,Tang等人\cite{tang2015influence}和Nguyen等人\cite{mtai2016sigmod}进一步提升了基于逆向可达集合的贪心算法的效果,主要是改进了分析的办法,利用更少的逆向可达集合来保证算法的运行。这一系列的影响力最大化相关工作大部分都是利用了次模性质来加速了近似算法,得到$1-\frac{1}{e}$的近似比。

种子集合最小化是和影响力最大化对立的问题,是找到一个影响力超过给定阈值的最小种子集合的问题。Ning Chen在08年\cite{Chen2008approximability}给出了固定阈值模型下很强的不可近似结果。固定阈值模型也是通用阈值模型的特例,固定阈值模型的阈值函数是在阈值处有一个从0到1的突变。Goyal在12年\cite{goyal2012minimizing}提出了一个双向近似的种子集合最小化的贪心算法。最近,Zhang等人\cite{zhang2014prob}在KDD14上第一次提出了种子集合最小化的一个概率变种模型,主要是关注给定的种子集合的影响力能否以一定的概率保证达到给定阈值,而之前的工作大部分只关心最终影响力的期望的大小。


非次模问题的最优化是近些年研究的另一个关注点,次模问题的最优化已经有比较成熟的结果,非次模问题还有很多需要探索。Du等人\cite{du2008analysis}在08年提出了约束次模和偏移次模两个技术用来研究非次模优化的贪心算法近似比,并且在Steiner Tree问题上给了细致的分析。Horel等人\cite{Horel2016sub}刚刚在NIPS16上研究了一类非次模但是和次模函数很接近的函数优化问题,他们假设函数值可以从一个数据库里面不断获取,关注的是找到函数h最大值点的同时需要从数据库询问的次数。

对于Kleinberg小世界网络\cite{Kleinberg2001small},Nguyen和Martel在2004年\cite{Martel2004analyzing}和2005年\cite{Nguyen2005analyzing}分析了小世界网络中简单传播时间的上下界,也就是小世界网络的半径。他们指出对于二维网格在$\alpha<4$时,Kleinberg’s model都是一个小世界网络,网络中的任意两个节点之间的最短路径是网络中节点个数的对数级。Ghasemiesfeh等人\cite{Ghasemiesfeh2013complex}在2013年首次给出了复杂传播的理论分析,他们研究了一个节点只有在至少$k$个邻居被激活时才能激活的$k$-复杂传播。他们的理论分析支持了弱连接在复杂传播过程中作用较小的观点,同时指出复杂传播的速度很大程度上取决于弱连接的分布,在15年\cite{ebrahimi2015complex}他们和Schoenebeck一起进一步探索了复杂传播和聚集系数$\alpha$的关系。Gao等人在16年\cite{gao2016gt}进一步研究了阈值模型下每个节点的阈值服从一个分布时$k$-复杂传播的一些问题。




\section{本文所做的工作}
社交网络影响力传播的非次模现象很普遍,但是非次模模型下影响力最大化问题的算法和不可近似性都很少有人讨论。
本论文研究通用阈值模型里两类比较特殊的非次模阈值函数并且深入下去。
本论文首先提出了$\varepsilon$-次模逼近和$k$-激活这两个非次模阈值函数,
然后分别证明了$\varepsilon$-次模逼近模型下影响力最大化难以近似和$k$-激活模型下复杂路由速度很慢。
最后在非次模影响力模型上设计了近似算法,研究了$\varepsilon$-次模逼近节点较少时的影响力最大化近似算法,并且设计了模拟实验。
实验结果显示我们提出的近似算法可以在$\varepsilon$-次模逼近节点较多时仍然表现很好。
本论文从一些角度探索了非次模影响力最大化问题,给出很强的不可近似性结果,也涉及了非次模节点个数较少时的近似算法,
对小世界网络$k$-激活路由的研究也填补了信息扩散和路由的最后一块空缺。
主要工作成果如下:


\begin{enumerate}
\item  提出$\varepsilon$-次模逼近函数和$k$-激活函数。$\varepsilon$-次模逼近函数是一类存在次模上界$f^{sub}$和下界$f^{sub}(1-\varepsilon)$的函数,而且上下界只在比例上相差$\varepsilon$。这类函数和次模函数很接近,但又不具有次模函数的性质。$k$-激活函数对应着一个节点只有在其$k \geq 2$ 个邻居都被激活后自己才会被激活的传播模型。
\item  通过构建机关(gadgets)和NP完全归约,证明了$n$个节点的图中图中有$n^{\gamma}$个$\varepsilon$-次模逼近节点时,影响力最大化问题很难被近似。
通用阈值模型在每个节点的阈值函数都是次模的时候,影响力函数也是次模的\cite{Mossel2007sub}。$\varepsilon$-次模逼近函数不是次模,但又和次模函数相距很近,但是本文证明了$n^{\gamma}$个$\varepsilon$-次模逼近节点足以让整体影响力函数偏移次模很远,阈值函数很小的偏移会导致整体影响力函数不可近似。
而在$\varepsilon$-次模逼近节点较少的图中,我们利用$\varepsilon$-次模逼近函数的次模上下界,设计了贪心的近似算法。
然后基于概率空间的映射,证明了图中有$\ell$个$\varepsilon$-次模逼近节点时,算法可以取得$(1-\varepsilon)^{\ell}(1-\frac{1}{e})$的近似比。
\item  通过对Kleinberg小世界网络中弱连接分布的深入研究,证明了$n$个节点小世界网络中,对于任意范围内的聚集系数$\alpha$,$k$-激活路由(复杂路由)时间是$n$的多项式的。即使在简单路由和复杂传播速度都很快的$\alpha=2$处,$k$-激活路由仍然需要$\Omega(n^{\frac{1}{4}})$步才能激活目标节点。
对比简单传播、简单路由和复杂传播较快的速度($n$的对数多项式级),我们的结果显示复杂路由更加困难。
\item  然后借助蒙特卡罗模拟,设计实验模拟影响力传播过程。实现了本文提出的算法和几种基准算法,发现在所有真实网络数据集合上本文提出的算法均表现最好。
甚至在网络中$\varepsilon$-次模逼近节点个数较多时,算法依然比贪心算法略好。
随着网络中$\varepsilon$-次模逼近节点个数的增加,$\varepsilon$的增加,本文提出的算法在效果上领先基准算法越多。
\end{enumerate}

\section{论文组织结构}
论文组织结构如下:

第一章~绪论

介绍课题的研究背景,阐述了影响力最大化和小世界现象的研究背景以及国内外研究现状,提出本文的研究目标和主要工作内容。

第二章~小世界网络及影响力传播基础

介绍本文工作涉及到的主要相关理论及模型,重点分析了Kleinberg小世界网络以及影响力传播模型,
然后介绍了影响力最大化的相关算法。

第三章~非次模影响力传播模型

主要介绍了本文后续降采用的阈值函数和相关问题定义,定义了$\varepsilon$-次模逼近函数、$k$-激活函数和本文研究的问题。

第四章~$\varepsilon$-次模逼近网络的影响力最大化

主要讨论了$\varepsilon$-次模逼近网络中影响力最大化问题,
给出了图中有$n$的多项式个$\varepsilon$-次模逼近节点时的不可近似性,
最后针对较少$\varepsilon$-次模逼近节点的情况,
分别基于贪心策略和概率空间映射策略设计了两个近似算法,各自分析了近似比。


第五章~$k$-激活路由的研究

给出$k$-激活路由的结论及完整的证明过程,然后探讨一步允许感染$m$个节点的复杂路由,建立起$k$-激活路由和$k$-激活传播的关系。


第六章~实验与分析

实现了第四章提出的近似算法\textsf{Galg-U}和\textsf{Galg-L},
并且在真实的社交网络数据{\em NetHEPT}、{\em Flixster}和{\em DBLP}上测试对比了算法和其他基准影响力最大化算法效果并分析了实验结果。

第七章~总结与展望

在最后的总结与展望中,首先总结了本文的工作。
然后指出已有工作的不足,并对下一步的工作进行展望。