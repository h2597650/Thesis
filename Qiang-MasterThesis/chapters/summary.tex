\chapter{总结与展望}
\section{研究工作总结}
信息化社会的到来,社交网络的用户量在爆发式增长。
社交网络中信息的扩散机制和基于社交网络的营销策略越发成为研究热点。
在本文中,我们研究学习了社交网络中一些非次模的问题,主要是基于非次模的阈值函数。
之前的研究大多集中在次模阈值函数和次模影响力函数,我们的研究和他们大不相同。
根据前人的实验观察,我们研究了两类非次模函数----$\varepsilon$-次模逼近函数和$k$-激活函数,
然后分别讨论了基于$\varepsilon$-次模逼近函数的影响力最大化问题和基于$k$-激活函数的小世界网络路由问题。
首先我们证明了尽管$\varepsilon$-次模逼近函数和次模函数很接近,
在有$\varepsilon$-次模逼近节点的图中影响力最大化问题依然很难近似。
通过构造概率与门并从NP完全问题集合覆盖规约,我们证明了
$n$个节点的图中只要有$n$的多项式个$\varepsilon$-次模逼近节点影响力最大化算法就无法做到$log$近似。
然后我们通过把$\varepsilon$-次模逼近函数替换成次模的上界或者下界,转化问一个次模优化问题,
设计了近似算法\textsf{Galg-U}和\textsf{Galg-L}。
基于概率空间映射,我们证明了这个近似算法的近似比为$(1-\frac{1}{e})(1-\varepsilon)^c$,
其中$c$是$\varepsilon$-次模逼近节点的个数。
接下来我们研究了Kleinberg小世界网络中,基于$k$-激活函数的路由。
本文定量地研究了从两个相邻的种子节点开始去感染一个网格上最远目标$t$需要的步数,也就是路由时间。
在每一步,只有一个节点会被激活,选择被激活节点的策略是分散式的,也就是路由选择策略只能基于当前被激活的节点发出的强连接和弱连接。
复杂路由比较像社交网站上最近提出的一个应用:主动交友,主动交友是指在像Facebook、人人网等社交网站上通过添加一些中间好友来增加目标接受自己好友请求的概率\cite{YangHLC13}。
本文指出与$k$-激活传播不同,对于所有的$\alpha$,$k$-激活路由时间均有$n$的多项式的下界,即使每一步允许激活多个节点。


本文的最后实现了算法\textsf{Galg-U}和\textsf{Galg-L}和其他基准算法,
并且在真实的社交网络{\em NetHEPT}、{\em Flixster}和{\em DBLP}上测试对比了算法和其他基准算法效果。
在小数据集上,\textsf{Galg-U}和\textsf{Galg-L}算法效果比\textsf{Greedy}稍好,
但是算法的运行时间要少很多,因为\textsf{Galg-U}和\textsf{Galg-L}可以利用一些次模上下界利用\textsf{TIM}算法加速。
在较大的网络{\em Flixster}和{\em DBLP}上,\textsf{Galg-U}和\textsf{Galg-L}均比其他基准算法要好。
而且在$\varepsilon$-次模逼近节点个数达到总结点个数$\frac{1}{3}$时,算法的表现依然很好。
这说明\textsf{Galg-U}和\textsf{Galg-L}不仅仅有理论近似比保证,实际中运行效果也很好。




\section{未来工作的展望}
在影响力最大化问题上,本文目前的研究主要集中在$\varepsilon$-次模逼近函数,
但是对于那些和次模函数相距很远的非次模阈值函数,我们还不知道如何设计有效的算法。
此外,我们提出的算法在实验中采用了逆向可达集合进行加速。
并不是所有的次模函数都可以使用逆向可达集合,
研究哪些函数可以使用逆向可达集合进行加速或者设计基于次模函数的加速算法也是新的研究方向。

其次,在研究$k$-激活路由时,为了更好地应用延迟选择原则,本文采用了有向Kleinberg小世界网络模型。
因为每个节点的弱连接平均数量仍为常数,有向小世界网络和无向小世界网络中$k$-激活路由的效率应该不会相差太多。
后续的工作希望可以研究$k$-激活路由在无向Kleinberg小世界网络中的性能。
未来也可以研究在什么样结构的网络中$k$-激活路由可以很快找到目标,探讨$k$-激活路由在其他小世界模型中的执行效率。
为$k$-激活路由寻找更广泛的应用场景也是未来需要进行的工作之一。


