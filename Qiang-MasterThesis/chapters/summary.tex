\chapter{总结与展望}
\section{研究工作总结}
随着互联网和多媒体技术的普及,相似图像检索测逐渐成为一个研究热点。在基于BoW的检索框架\cite{sivic2003video}被提出后,CBIR得到了质的发展。在这之后每年都有针对BoW图像检索框架提出的改进方案,这些改进方案也涵盖了图像检索中的每一个环节。几何校验(geometric verification)则是这些改进方案中最受关注的一个方面。这是由于BoW框架自身存在的缺点:忽略了视觉特征之间的几何关系,这包括空间位置、尺度、角度等。对视觉特征进行几何校验一直被认为是提高BoW检索精度的关键问题。通常几何校验可以分为检索时校验和检索后校验。检索时校验有校验速度快、召回率高的特点;而检索后校验可以进行严格的一致性检测,可以保证较高的准确率。目前,在多个标准图像检索数据集上(Holidays、Oxford、Paris等),基于BoW的检索框架都达到了最高的检索精度和检索效率。近几年,深度卷积神经网在计算机视觉领域取得巨大的成功。由于CNN可以更好的理解图片的内容,克服图像检索中最大的困难——语义鸿沟,CNN被认为有潜力进一步提高CBIR的性能。由于基于CNN的图像检索方法还不成熟,很多基本的问题还亟待解决,如图片旋转的问题。目前解决CNN图像检索中图片旋转的方法都相对简单,在效果和扩展性上不能令人满意。

为了进一步提升CBIR的效果,我们针对不同的图像检索框架,对几何校验算法进行了研究。在基于BoW的图像检索框架中,我们分析了目前几何校验算法的优缺点,提出了Region Similarity Arrangement(RSA)几何校验算法。不同于传统的几何校验算法,关注视觉特征的空间位置或尺度、角度的相对变化,RSA试图分析一张图片中所有特征区域的属性的分布情况。为了达到这个目的,我们首先提出了Region Property Space(RPS),并将图片空间的每个视觉特征都映射到RPS中。在RPS中,一张图片的视觉特征属性的分布显而易见,并且可以简单的编码到BoW向量中,完成检索时校验。我们还对RPS中点的分布进行了细致的分析,提出了Spatial Weighting(SpW),SpW可以很好的解决图像检索中的burstiness问题。同时,针对这种特殊的分布情况,我们还提出了计算RSA向量的快速算法。通过量化,RSA不增加倒排表的内存,并且计算复杂度极低,适用于大规模图像检索问题。针对基于CNN的图像检索中存在的问题,我们提出了MONet用来学习图片的主方向。MONet识别图片主方向的准确率在80\%以上,可以很好的解决CNN图像检索中查询图片旋转的问题。并且MONet可以与多种现现有的框架结合,具有很好的扩展性。我们介绍了MONet的两个应用场景,分别为在传统的BoW检索框架下和民族服饰识别问题中MONet的应用。

\section{未来工作的展望}
本文提出的RSA几何校验和MONet还存在一些局限性。RSA只考虑了特征区域的尺度的角度信息,没有考虑其在图片中间中的位置信息;RSA对旋转变化敏感,即使旋转的RSA向量也不能很好的解决这一问题。MONet目前只能处理四种图片主方向,且准确率还需要进一步提高;使用MONet进行预处理不能形成端到端的模型,检索速度会下降。

在未来的工作中,我们希望可以不断完善基于CNN的检索框架,借鉴BoW模型中的各种经典算法,使基于CNN的图像检索可以达到更好的准确率。针对几何校验问题,我们希望可以提出更加快速并且校验力度强的算法,同时考虑特征区域的属性信息和位置信息,并将检索时校验和检索后校验结合,完成对图片局部和全局几何一致性的度量。对于MONet,我们期望可以收集更多的数据,设计更加合理的网络结构,可以让网络对图片的主方向更加敏感,并完成更加细粒度的图片主方向判定。

